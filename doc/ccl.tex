\documentclass[a4paper,11pt,parskip=half]{scrartcl}
\usepackage[utf8]{inputenc}
\usepackage[T1]{fontenc}
\usepackage[colorlinks=true]{hyperref}
\usepackage[textsize=footnotesize]{todonotes}
\usepackage{amsmath,amssymb,mathtools,cleveref,multirow,stmaryrd}

\title{CCL -- compiling Clerical}
\author{Franz Brauße\footnote{%
	University of Trier, 54286 Trier;
	E-Mail: \href{mailto:brausse@informatik.uni-trier.de}%
	                   {brausse@informatik.uni-trier.de};
	supported by the German Research Foundation (DFG),
	project WERA, grant MU 1801/5-1}}

\newcommand*\iRRAM{\texttt{iRRAM}}
\newcommand*\lang[1]{\text`\textcolor{blue}{\text{\ttfamily{#1}}}\text'{}}
\newcommand*\optional[1]{#1\ensuremath{{}_{\textrm{opt}}}}

\newcommand*\mto{\rightrightarrows}
%\newcommand*\pto{\rightharpoonup}
% requires mathabx, but that changes all common math symbols
%\newcommand*\pmto{\rightrightharpoons}
\newcommand*\dom{\operatorname{dom}}

% \ctxval{\gamma}{e} is the value of e evaluated in context \gamma
% \ctxval{e} is the value of e evaluated in context \gamma
\newcommand*\ctxval[2][\gamma]{\llbracket{#2}\rrbracket_{#1}}
\newcommand*\sctype[2][\sigma]{\operatorname{type}_{#1}(#2)}
\newcommand*\scro{\mathtt{ro}}
\newcommand*\scrw{\mathtt{rw}}

\newenvironment{notyet}{\color{violet}}{\color{black}}

\begin{document}
\maketitle
\begin{abstract}
CCL is a compiler for the language Clerical targeting the \iRRAM, Kirk and
possibly other ERC implementations. This document is a description of the
precise language and CCL's features.
\end{abstract}

\section{Concepts}
\subsection{Expressions}
The language is constructed by means of the grammar summarized in
\cref{fig:grammar}. For the complete grammar refer to the file
\texttt{src/cclerical.y}.
\begin{figure}
% cut -c7- grammar | sed -r 's#\$?@[0-9]+(:.*)?##g;s#"([^"]*)"|'"'([^']*)'"'#\\lang{\1\2}#g;s#\{\^\}#{\\^{}}#g;s#(^|\}|\s)([a-zA-Z_0-9]+)#\1\\emph\{\2\}#g;s#([$%_&])#\\\1#g;s#^([^ \t]+):#\1\\coloneqq\{\}\&#;s#\s+\|#\t|~\&#;s#\}(\s*[\\{])#\}~\1#g;' | cat -s | sed -r 's#(.)$#\1 \\\\#;s#^$#%#' | xclip -i
% then modified:
% - remove pure_expr
% - collect binary operators
% - use regexp to collapse lists:
%   - fun_call_param_spec, fun_call_params -> IDENT()
%   - tu, prog
%   - cases -> 'case' expr
%   - var_init_list -> 'var' expr
%   - {ext_,}fun_decl_params{,_spec} -> fun_decl
% - add IDENT
\begin{align*}
\emph{tu}~\coloneqq{}& \emph{toplevel}^* \\
%
\emph{toplevel}~\coloneqq{}& \emph{fun\_decl} \\
	|~& \lang{do}~ \emph{prog} \\
%
\emph{fun\_decl}~\coloneqq{}& \lang{function}~ \emph{IDENT}~  \lang{(}~ \optional{(\emph{type}~ \emph{IDENT}~ (\lang{,}~ \emph{type}~ \emph{IDENT})^*)}~ \lang{)}~ \lang{:}~ \emph{prog} \\
	|~& \lang{external}~ \emph{IDENT}~ \lang{(}~ \optional{(\emph{type}~ (\lang{,}~ \emph{type})^*)}~ \lang{)}~ \lang{->}~ \emph{utype} \\
%
\emph{prog}~\coloneqq{}& \emph{expr}~(\lang{;}~ \emph{expr})^* \\
%
\emph{expr}~\coloneqq{}& \emph{expr}~ \emph{bin\_op}~ \emph{expr} \\
	|~& \lang{!}~ \emph{expr} \\
	|~& \lang{-}~ \emph{expr} \\
	|~& \emph{IDENT}~ \lang{(}~ \optional{(\emph{expr}~(\lang,~\emph{expr})^*)}~ \lang{)} \\
	|~& \emph{IDENT} \\
	|~& \emph{CONSTANT} \\
	|~& \lang{lim}~ \emph{IDENT}~  \lang{=>}~ \emph{expr} \\
	|~& \lang{(}~ \emph{prog}~ \lang{)} \\
	|~& \lang{case}~ \emph{expr}~ \lang{=>}~ \emph{expr}~ (\lang{||}~ \emph{expr}~ \lang{=>}~ \emph{expr})^*~ \lang{end} \\
	|~& \lang{if}~ \emph{expr}~ \lang{then}~ \emph{expr}~ \optional{(\lang{else}~ \emph{expr})} \\
	|~& \lang{var}~  \emph{var\_init}~ (\lang{and}~ \emph{var\_init})^*~ \lang{in}~ \emph{expr} \\
	|~& \emph{IDENT}~ \lang{:=}~ \emph{expr} \\
	|~& \lang{skip} \\
	|~& \lang{while}~ \emph{expr}~ \lang{do}~ \emph{expr} \\
%
\emph{var\_init}~\coloneqq{}& \emph{IDENT}~ \lang{:=}~ \emph{expr}
	\begin{notyet}~\optional{(\lang:~\emph{type})}\end{notyet}
	\\
%
\emph{bin\_op}~\coloneqq{}&
	\lang{+}~|~\lang{-}~|~\lang{*}~|~\lang{/}~|~\lang{\^{}}~|~\lang{<}~
	|~\lang{>}~|~\lang{|}~|~\lang{\&}~|~\lang{/=}
	\begin{notyet}~|~\lang{<=}~|~\lang{>=}~|~\lang{==}\end{notyet}
	\\
%
\emph{type}~\coloneqq{}& \lang{Bool}~|~ \lang{Int}~|~ \lang{Real} \\
%
\emph{utype}~\coloneqq{}& \emph{type}~|~ \lang{Unit} \\
%
\emph{IDENT}~\coloneqq{}& [\texttt{a-z\_}][\texttt{a-zA-Z0-9\_}]^*\texttt'^* \\
%
\emph{CONSTANT}~\coloneqq{}& \emph{NAT}~|~\emph{FRAC}
\end{align*}
\caption{Summarized grammar of CCL; the specification of the usual precedence
	and associativity of operators and \lang{then}/\lang{else} as well as
	the format of the literal constants \emph{NAT} and \emph{FRAC} has been
	omitted.}
\label{fig:grammar}
\end{figure}


\subsection{Types}
Clerical has 3 built-in named types (\cref{fig:named-types}) in addition to --
for sake of reasoning about syntactical constructs --
two unnamed types (\cref{fig:unnamed-types}). The types \lang{Integer}%
\begin{notyet}, \lang{Rational}\end{notyet} and
\lang{Real} belong to the set of \emph{arithmetical types}. The types
\lang{Boolean} and \lang{Kleenean} form the set of \emph{logical types}.

Let
\[ \dom(\tau)=\begin{cases}
	\{\lang{False},\lang{True}\},&\text{if}~\tau=\lang{Boolean} \\
	\{\lang{False},\lang{True},\bot\},&\text{if}~\tau=\lang{Kleenean} \\
	\mathbb Z,&\text{if}~\tau=\lang{Integer} \\
\begin{notyet}
	\mathbb Q,
\end{notyet} &
\begin{notyet}
	\text{if}~\tau=\lang{Rational}
\end{notyet} \\
	\mathbb R,&\text{if}~\tau=\lang{Real}
\end{cases} \]

The set of primitive types is
$\overline T=\{\lang{Boolean},\lang{Kleenean},\lang{Integer},
\begin{notyet}
\lang{Rational},
\end{notyet}\lang{Real}\}$.

The set of all types is $T=C\cup F$ where $C$ is called the set
of \emph{compound types} and $F$ is the set of \emph{function types}.
Both are constructed recursively:
\begin{itemize}
\item $\overline T\subseteq C$.
\begin{notyet}
\item $A\in C^+\implies A\in C$ and $\dom(A)=\bigtimes_{i=1,\ldots,|A|} A_i$
\item $A,B\in T\implies \{A\to B,{\subseteq}A\to B,A\mto B,{\subseteq}A\mto B\}\subseteq F$.
\end{notyet}
\end{itemize}
The empty word over $C$ is called $\texttt{Unit}$ and
$\dom(\texttt{Unit})=\{\star\}$.

\subsection{Context}
A \emph{scope} is a finite sequence $\sigma\in(I\times T\times\{\scro,\scrw\})^*$
where $I$ are \emph{identifiers}, i.e.\ words over $\Sigma$,
and $T$ is the set of types. $\scro$ and $\scrw$ specify whether a location is
read-only and read-writable, respectively.
Let $\sigma_I^{-1}(x)=\{i:\sigma_i=(x,\tau),\tau\in T\}$ denote the set of
indices of an identifier $x\in I$ in $\sigma$.

Given a scope $\sigma=(x_i,\tau_i,r_i)_{i=1,\ldots,k}$,
the set of all \emph{contexts valid on $\sigma$} is
\[ \Gamma_\sigma
  =\{(v_i)_{i=1,\ldots,k}:v_i\in\dom(\tau_i)~\text{for}~i\in\{1,\ldots,k\}\}
  \text.
\]

In order to define semantics of the constructs of the language,
in the following, given a scope $\sigma$, a well-typed expression $e$
interpreted in a context $\gamma\in\Gamma_\sigma$ has a value of the type of $e$,
in symbols $\ctxval e\in\dom(\sctype e)$. Only assignments $c$
modify contexts, $\gamma(c)\in\Gamma_\sigma$;
for all other expressions $d$ it holds that $\gamma(d)=\gamma$.

% TODO: it is not *often* necessary, just for assignment; move to assignment
In the following it is often necessary to replace a value $\gamma_k$ in a
context $\gamma\in\Gamma_\sigma$ with a value $v$ where $k$ refers to the last
declaration of a variable with identifier $x$.
We thus define the shortcut notation
$\gamma[x/v]\coloneqq
(\gamma_1,\ldots,\gamma_{k-1},v,\gamma_{k+1},\ldots,\gamma_{|\sigma|})$
if $x$ is an identifier, $\sigma_I^{-1}(x)\neq\varnothing$,
$k\coloneqq\max\sigma_I^{-1}(x)$ and $\sigma_k=(x,\operatorname{type}(v),\scrw)$,
otherwise $\gamma[x/v]$ is undefined.

An expression $e$ is \emph{pure} if for all contexts $\gamma$ s.t.\ $e$ is valid
in $\gamma$, $\gamma(e)=\gamma$.

\subsection{Variables}
Variables identified by $x\in\Sigma^*$ declared in a scope $\sigma$ are typed,
i.e.\ $\operatorname{type}_\sigma(v)\in T$.

\subsection{Type conversions}
There are no implicit type conversions. Explicit conversions between
\emph{compatible types} are provided as functions in the standard library.

\subsection{Operators}
In general, only assignment $x~\lang{:=}~e$ and expressions containing it are
im-pure expressions.
Specifically for function calls, the arguments must be pure expressions as
must be the operands to all operators.
% In CCL we can easily check this through the type of an expression.

\subsubsection{Arithmetical operators}
Unary operators $\lang+,\lang-:A\to A$ and
binary operators $\lang+,\lang-,\lang*,\lang/,\lang{\^{}}:A\times A\to A$
are built-in for arithmetical types $A$.
$\lang/$ on \lang{Integer} truncates the result.

Formally, for a scope $\sigma$ and well-typed expressions $x,y$
an expression $c_\theta\coloneqq x\mathbin\theta y$ with
$\theta\in\{\lang+,\lang-,\lang*,\lang/,\lang{\^{}}\}$ is \emph{well-typed}
if the operands have the same arithmetical type
$\tau\in T$ in $\sigma$,
i.e.\ $\operatorname{type}_\sigma(x)=\tau=\operatorname{type}_\sigma(y)$.
Then $\operatorname{type}_\sigma(c_\theta)=\tau$ and -- if $x,y$ are pure --, given
a context $\gamma\in\Gamma_\sigma$,
\[ \ctxval{c_\theta}\coloneqq\begin{cases}
      \ctxval x+\ctxval y,&\text{if}~\theta=\lang+ \\
      \ctxval x-\ctxval y,&\text{if}~\theta=\lang- \\
      \ctxval x\cdot\ctxval y,&\text{if}~\theta=\lang* \\
      \operatorname{trunc}(\ctxval x/\ctxval y),&\text{if}~\theta=\lang/~\text{and}~\tau=\lang{Integer} \\
      \ctxval x/\ctxval y,&\text{if}~\theta=\lang/~\text{and}~\tau\neq\lang{Integer} \\
      \ctxval x^{\ctxval y},&\text{if}~\theta=\lang{\^{}}
   \end{cases} \]
where $\operatorname{trunc}:n\mapsto\operatorname{sgn}(n)\cdot\lfloor|n|\rfloor$.

\subsubsection{Logical operators}
Unary operator $\lang!:B\to B$ and
binary operators $\lang\&,\lang|,\lang{\^{}}:B\times B\to B$
are built-in for logical types $B$, i.e.\ the expressions
$c\in\{\lang!x,x\mathbin\theta y\}$
for $\theta\in\{\lang\&,\lang|,\lang{\^{}}\}$ and expressions $x,y$
in a scope $\sigma$ are well-typed if $\sctype x=\sctype y\in B$.
Then $\sctype c\coloneqq\sctype x$. Given a context $\gamma\in\Gamma_\sigma$, 
For $\sctype c=\lang{Boolean}$, $\ctxval c$ results from classical logic.
For $\sctype c=\lang{Kleenean}$, $\ctxval c$ results from Kleene's
three-valued logic and is summarized in \cref{tab:kleene-ops}.

\subsubsection{Comparison operators}
There are built-in operators
$\lang{==},\lang{!=}:\overline T\times\overline  T\to R$
for primitive types $\overline T$
and $R$ according to \cref{tab:cmp-ops-types}.

\subsection{Declarations}
\subsubsection{Variable declarations}
New variables are introduced by declaring and initializing them and they are
valid within an expression. After that the \emph{scope} of the variable is left
and it is no more accessible. Syntax:
\begin{align*}
	\emph{var-decl}\coloneqq{}&
		\lang{var}~\emph{ident}~\lang{:=}~\emph{expr}
		~\optional{\emph{var-decl-and-clause}}~\lang{in}~\emph{expr} \\
	\emph{var-decl-and-clause}\coloneqq{}&
		\lang{and}~\emph{ident}~\lang{:=}~\emph{expr}
		~\optional{\emph{var-decl-and-clause}}
\end{align*}
The sequence
\[\lang{var}~x_1~\lang{:=}~e_1~\lang{and}~x_2~\lang{:=}~e_2~\lang{in}~e\]
is equivalent to
\[\lang{var}~x_1~\lang{:=}~e_1~\lang{in}~\lang(
~\lang{var}~x_2~\lang{:=}~e_2~\lang{in}~e~\lang)\text.\]

Given a scope $\sigma$, an identifier $x$
and a pure expression $e$ with $\sctype e\neq\lang{Unit}$, then the expression
$c\coloneqq\lang{var}~x~\lang{:=}~e~\lang{in}~f$ is well-typed and
$\sctype c\coloneqq\sctype[\sigma'] f$
where $\sigma'\coloneqq(\sigma_1,\ldots,\sigma_{|\sigma|},(x,\sctype e,\scrw))$.
If $\gamma\in\Gamma_\sigma$ is a context, let
$\gamma'\coloneqq(\gamma_1,\ldots,\gamma_{|\sigma|},\ctxval e)\in\Gamma_{\sigma'}$,
then $\ctxval c\coloneqq\ctxval[\gamma'] f$.

\subsubsection{Function declarations}
\begin{itemize}
\item
	External functions are provided in form of a \emph{library}, that is,
	a precompiled object file containing symbols corresponding to the
	function described. Syntax:
	\begin{align*}
		\emph{ext-fun-decl}\coloneqq{}&
			\lang{external}~\emph{ident}
			~\lang(~\optional{\emph{ext-argument-list}}~\lang)
			~\lang{->}~\emph{type} \\
		\emph{ext-argument-list}\coloneqq{}&
			\emph{type}~\optional{\emph{ident}}
			~(\lang,~\emph{type}~\optional{\emph{ident}})^*
	\end{align*}
\item
	(Internal) function declarations attach code to an identifier with
	parameters. Syntax:
	\begin{align*}
		\emph{fun-decl}\coloneqq{}&
			\lang{function}~\emph{ident}
			~\lang(~\optional{\emph{argument-list}}~\lang)
			~\optional{(\lang{->}~\emph{type})}~\lang:~\emph{expr} \\
		\emph{argument-list}\coloneqq{}&
			\emph{type}~\emph{ident}~(\lang,~\emph{type}~\emph{ident})^*
	\end{align*}
\end{itemize}

\subsection{Core language constructs}
\subsubsection{Assignment}
Syntax: $\emph{ident}~\lang{:=}~\emph{expr}$

Let $c\coloneqq x~\lang{:=}~e$ in context $\gamma\in\Gamma_\sigma$.
If $e$ is a pure expression
and $\sigma_I^{-1}(x)\neq\varnothing$, let $j=\max\sigma_I^{-1}(x)$ and
if $\operatorname{type}(e)=\tau$ where $(x,\tau,\scrw)=\sigma_j$, then
$\gamma(c)\coloneqq\gamma[x/\operatorname{value}(e)]$.

\subsubsection{Function calls}
Syntax:
$\emph{ident}~\lang(~(\emph{expr}~\optional{(\lang,~\emph{expr})^*)}~\lang)$

Let $c\coloneqq f~\lang(~e_1\lang,~\ldots\lang,~e_n~\lang)$.

A function's body can only refer to (global) identifiers fully defined
previously, thus preventing recursion.

\subsubsection{Loops}
Syntax: $\lang{while}~\emph{expr}~\lang{do}~\emph{expr}$

Let $c\coloneqq\lang{while}~e~\lang{do}~f$.
If $e$ is a \emph{pure} expression with $\operatorname{type}(e)=\texttt{Bool}$
and $\operatorname{type}(f)=\texttt{Unit}$
then $\operatorname{type}(c)=\texttt{Unit}$.

\subsubsection{Limits}
Syntax: $\lang{lim}~\emph{ident}~\lang{=>}~\emph{expr}$

Let $c\coloneqq\lang{lim}~n~\lang{=>}~e$ and $\sigma$ be a scope.
Then $c$ is well-typed if $e$ is a pure expression in
$\sigma'\coloneqq(\sigma_1,\ldots,\sigma_{|\sigma|},(n,\lang{Int},\scro))$.
Let $\gamma\in\Gamma_\sigma$ be a context and
let $s_n\coloneqq\ctxval[(\gamma_1,\ldots,\gamma_{|\sigma|},n)] e$ be the
value of the expression $e$ for a fixed $n\in\mathbb N$. If
$\forall n,m\in\mathbb N:|s_n-s_m|\leq2^{-n}+2^{-m}$
then $\ctxval c\coloneqq\lim_{n\to\infty} s_n$.

\subsubsection{Multi-valued choice}
Syntax:
$\lang{case}~(\optional{\lang{||}}~\emph{expr}~\lang{=>}~\emph{expr})^*
~\lang{end}$

Let $c\coloneqq\lang{case}~(\lang{||}~e_i~\lang{=>}~f_i)_{i=1,\ldots,n}~\lang{end}$.
If $\operatorname{type}(e_i)\in\{\texttt{Boolean},\texttt{Kleenean}\}$
and $\operatorname{type}(f_i)=\tau$ for all $i=1,\ldots,n$
then $\operatorname{type}(c)\coloneqq\tau$ and
$\operatorname{value}(c)\in\{\operatorname{value}(f_i):
\operatorname{value}(e_i)=\lang{True},i\in\{1,\ldots,n\}\}$
holds.


\subsection{Syntactical sugar}
\subsubsection{If-then-else}
\begin{itemize}
\item
	$\lang{if}~b~\lang{then}~c$ corresponds to
	$\lang{case}~b~\lang{==}~\lang{True}~\lang{=>}~c~\lang{end}$.
\item
	$\lang{if}~b~\lang{then}~c_1~\lang{else}~c_2$ corresponds to
	$\lang{case}~b~\lang{==}
	~\lang{False}~\lang{=>}~c_2~\lang{||}
	~b~\lang{==}~\lang{True}~\lang{=>}~c_1~\lang{end}$.
\end{itemize}

\subsection{Standard Library}
The CCL standard library provides the following functions.
\begin{itemize}
\item Type conversion $\mathbb Z\to\mathbb R$:
	$\lang{external real(Integer) -> Real}$
%\item $\mathbb Q\to\mathbb R$: $\lang{external real(Rational) -> Real}$
%\item $\mathbb Z\to\mathbb Q$: $\lang{external rational(Integer) -> Rational}$
\item Absolute value: $\lang{external abs(Real) -> Real}$
\item Multi-valued boundedness test: $\lang{external bounded(Real, Integer) -> Boolean}$
\end{itemize}


\section{Syntax}
\subsection{Grammar}


\section{Semantics}

\newpage
\appendix
\section{}

\begin{figure}
\begin{description}
\item[\texttt{Bool}]
	Corresponds to the set $\mathbb B=\{0,1\}$.
	Named constants \lang{False} and \lang{True} correspond to $0$ and
	$1$, respectively.
\item[\texttt{Integer}]
	Corresponds to the set $\mathbb Z$. Literals matching one of the PCREs
	\[
		\verb#(0|[1-9][0-9]*)# \quad \verb#0[xX][0-9a-fA-F]+# \quad
		\verb#0[bB][0-1]+# \quad \verb#0(0|[1-7][0-9]*)#
	\]
	are interpreted as integers written in decimal, hexadecimal, binary or
	octal notation.
\item[\texttt{Real}]
	Corresponds to the set $\mathbb R$. Literals matching one of the PCREs
	\[ \verb#((0|[1-9][0-9]*)\.|\.[0-9])[0-9]*([eE][+-]?[0-9]+)?# \]
	\[ \verb#0[xX]([0-9a-fA-F]+\.|\.[0-9a-fA-F])[0-9a-fA-F]*([eE][+-]?[0-9a-fA-F]+)?# \]
	are interpreted as rational numbers in fractional decimal or hexadecimal
	notation.
\end{description}
\caption{Named types.}
\label{fig:named-types}
\end{figure}

\begin{figure}
\begin{description}
\item[\texttt{Unit}]
	Corresponds to the singleton set $\{\star\}$.
	There are no named constants.
\item[\texttt{Kleenean}]
	Corresponds to the set $\mathbb K=\{0,1,\bot\}$.
	There are no named constants.
\end{description}
\caption{Unnamed types.}
\label{fig:unnamed-types}
\end{figure}

\begin{table}
\centering
\begin{tabular}{c|c}
	$a$ & $\lang!a$ \\ \hline
	\lang{False} & \lang{True} \\
	\lang{True} & \lang{False} \\
	$\bot$ & $\bot$
\end{tabular}
\qquad
\begin{tabular}{c|ccc}
	\lang\& & \lang{False} & \lang{True} & $\bot$ \\ \hline
	\lang{False} & \lang{False} & \lang{False} & $\bot$ \\
	\lang{True}  & \lang{False} & \lang{True}  & $\bot$ \\
	$\bot$       & $\bot$       & $\bot$       & $\bot$ \\
\end{tabular}
\qquad
\begin{tabular}{c|ccc}
	\lang| & \lang{False} & \lang{True} & $\bot$ \\ \hline
	\lang{False} & \lang{False} & \lang{True}  & $\bot$ \\
	\lang{True}  & \lang{True}  & \lang{True}  & $\bot$ \\
	$\bot$       & $\bot$       & $\bot$       & $\bot$ \\
\end{tabular}
\caption{Truth-tables for logical operators on type \lang{Kleenean}.}
\label{tab:kleene-ops}
\end{table}

\begin{table}
\centering
\begin{tabular}{c|c}
	$T$ & $R$ \\ \hline
	\lang{Boolean} & \lang{Boolean} \\
	\lang{Kleenean} & \lang{Kleenean} \\
	\lang{Integer} & \lang{Boolean} \\
	\lang{Rational} & \lang{Boolean} \\
	\lang{Real} & \lang{Kleenean}
\end{tabular}
\caption{Combinations of argument types $T$ and return types $R$ of the in-built
	comparison operators $\lang{==},\lang{!=}:T\times T\to R$.}
\label{tab:cmp-ops-types}
\end{table}

\end{document}
